%%%%%%%%%%%%%%%%%%%%%%%%%%%%%%%%%%%%%%%%%

%
% Important note:
% This template requires the resume.cls file to be in the same directory as the
% .tex file. The resume.cls file provides the resume style used for structuring the
% document.

%
% Creator Peilin Li
% Contact me via twitter/wechat: @pe1l1nl1
% linkedin.com/peill and/or github/ppeill
% Inspired by Peppa Pig 
%%%%%%%%%%%%%%%%%%%%%%%%%%%%%%%%%%%%%%%%

%----------------------------------------------------------------------------------------
%	PACKAGES AND OTHER DOCUMENT CONFIGURATIONS
%----------------------------------------------------------------------------------------

\documentclass{resume} % Use the custom resume.cls style

\usepackage[left=0.40in,top=0.3in,right=0.75in,bottom=0.1in]{geometry} % Document margins
\usepackage{fontawesome}
\usepackage{times}
\newcommand{\tab}[1]{\hspace{.2667\textwidth}\rlap{#1}}
\newcommand{\itab}[1]{\hspace{0em}\rlap{#1}}

\usepackage[T2A]{fontenc}
\usepackage[utf8]{inputenc}
\usepackage[russian]{babel}
% \begin{center}

% \end{center}
\name{ILYA BARYSHNIKOV} % Your name 


%\address{123 Pleasant Lane \\ City, State 12345} % Your secondary addess (optional)
\address{\extrainfo{\faTelegram \href{https://t.me/zlonast}{ Zlonast}} \\ \extrainfo{\faGithub \href{https://github.com/zloilya}{ Zloilya}} \\ \href{mailto:zlonast3@gmail.com}{zlonast3@gmail.com}}
\address{\faMapMarker{ Saint Petersburg, Russsia}}

\begin{document}
{\centerline {\em \textbf { Looking for a full-time job (I would like to be remote), vacancy Haskell Junior Developer } } }
%----------------------------------------------------------------------------------------
%	EDUCATION SECTION
%----------------------------------------------------------------------------------------

\begin{rSection}{Education}

{\bf ITMO University, Saint Petersburg, Russsia} \hfill {\em Aug 2019 - Dec 2021} 
\\{ \textit {Bachelor of science in computer science }}  \hfill


{ \textit {Relevant coursework:}}  \hfill
\\Mathematical logic, Type Theory, Java, Lean3, Arend, C++,  Algorithms and Data Structures  \hfill
%Minor in Linguistics \smallskip \\
%Member of Eta Kappa Nu \\
%Member of Upsilon Pi Epsilon \\


\end{rSection}

\begin{rSection}{Skills}

% , Java
\begin{tabular}{ @{} >{\bfseries}l @{\hspace{6ex}} l }
Programming: \ & Haskell \\
Some experience : \ & JavaScript \\
Software \& Tools: & Git, Bash, SQL, Nix\\
Languages: \ & Russian(native), English(intermediate)
\end{tabular}

\end{rSection}

% \begin{rSection}{Carrier Objective}
%  To work for an organization which provides me the opportunity to improve my skills and knowledge to grow along with the organization objective.
% \end{rSection}
%--------------------------------------------------------------------------------
%    Projects And Seminars
%--------------------------------------------------------------------------------


\begin{rSection}{Haskell Projects (All available in Github)}

{\bf News server} \hfill {\em Apr 2021} 
\\{\textit{some libraries: wai, warp, cryptonite, postgresql-simple}}
\\-  Simple server can send news, can filter it, have som endpoints for admin, registration

{\bf Echo bot} \hfill {\em Mar 2021} 
\\{\textit{some libraries: http-client, yaml, postgresql-simple, aeson}}
\\-  Echo repeat n times for your message or sticker

{\bf Programming language Hi} \hfill {\em Jan 2021} \\
\textit{some libraries: megaparsec, prettyprinter, bytestring} \\
- In this homework we will gradually develop a small programming language
called Hi. site: \href{https://int-index.gitlab.io/hw-checker/#homework-3}{ task text} 

{\bf Telegram bot} \hfill {\em Dec 2021} 
\\{\textit{some libraries: servant, aeson}}
\\{\textit{ Some stuff to cross sending telegram channel and group}}
\\- Add comment for group and add channel for group (revers as usual)

{\bf Mixtape small Haskell homeworks} \hfill {\em Sep 2021 - Nov 2021} 
\\- Homeworks from the FP course organized by Serokell

{\bf Mathematical logic done with Haskell} \hfill {\em Jun 2021 - Jun 2021} 
%\\{\textit{ Some stuff to cross sending telegram channel and group}}
\\- In this problem, they want to check the proof of the expression in the Hilbert version of the intuitionistic propositional calculus and rebuild it into a proof in natural deduction.

%{\bf Web Crawler} \hfill {\em Apr 2021 - May 2021} 
%\\ { \scriptsize MULTI-THREADED JAVA WEB CRAWLER }
%\\- Created a thread-safe application that browses the World Wide Web by starting at the given url and extracting all the hyperlinks in the page to recursively visit next using Java Concurrency Utilities
%\\- Added a possibility to limit the number of page downloads, hyperlink extractions, and downloads per host within the same time.

%{\bf Bimap} \hfill {\em Oct 2020 - Dec 2020} 
%\\{\textit{ C++ Data Structure}}
%\\- Unlike map, bimap searches can be performed both on the left (left) elements of pairs, and on the right (right).

%{\bf Some small C++ homeworks} \hfill {\em Sep 2020 - Dec 2020} 
%\\- intrusive list, shared ptr

\end{rSection}

%----------------------------------------------------------------------------------------
%	Extracurricular Section

%----------------------------------------------------------------------------------------
\begin{rSection}{Haskell Books and Courses }

{\textbf {Have commits in http-client and telegram-bot-simple}}
\\
\\
{\textit{First and Second Haskell stepik complite}} \\
{\textit{Haskell in Depth}} \\
{\textit{ Practical Web Development with Haskell }} \\
{\textit{ Parallel and Concurrent Programming in Haskell}} \\
{\textit{ Purely Functional Data Structures (first 100 pages)}} \\

\end{rSection}
\begin{rSection}{Honors \& Awards}
- 2019  \textbf {Finalist}, Participated in the finals of Technocup Olympiad held by Mail.ru and MIPT \hfill Moscow, Russia \\
- 2019  \textbf {Finalist}, Open Programming Contest \hfill Moscow, Russia \\
\end{rSection}

\end{document}